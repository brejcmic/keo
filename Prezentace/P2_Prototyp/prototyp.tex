%Beamer class
\documentclass{beamer}

\usepackage[czech]{babel}
\usepackage[utf8]{inputenc}
\usepackage{fontenc}
\usepackage{tgheros}
\usepackage{array}
\usepackage{color}
\usepackage{hyperref}

\usetheme{Antibes}
\usecolortheme{crane}


\title[Realizace prototypu]{Realizace prototypu}
\subtitle[KEO] {Konstrukce a realizace elektronických obvodů}
\author[Brejcha]{\texorpdfstring{Michal Brejcha\newline\url{brejcmic@fel.cvut.cz}}{Michal Brejcha}}
\institute[CVUT]{ČVUT v Praze, FEL}
\date[Praha, 2018]{Praha, 2018}

%------------------------------------------------------------------------------
%Konstanty a definice
%------------------------------------------------------------------------------
\newtheorem{myDef}{}
\newcommand{\kicadVersion}{5.0.0.}

\begin{document}
%------------------------------------------------------------------------------
%Uvodni slajd
%------------------------------------------------------------------------------
\frame{\titlepage}

\begin{frame}
\frametitle{Obsah} 
\tableofcontents
\end{frame}

\AtBeginSection[]
{
  \begin{frame}
    \frametitle{Téma}
    \tableofcontents[currentsection]
  \end{frame}
}

%------------------------------------------------------------------------------
%Instalace KiCAD
%------------------------------------------------------------------------------
\section{\texorpdfstring{Instalace KiCAD}{Instalace Kicad}}
%------------------------------------------------------------------------------
	\begin{frame}
    \frametitle{Stažení návrhového systému KiCAD}
		
		\begin{description}
			\item[url:] http://kicad-pcb.org/
			\item[sekce:] download
		\end{description}
		
		\begin{center}
			\includegraphics[scale=0.3]{obr/kicad_url.png}
		\end{center}
	\end{frame}
%------------------------------------------------------------------------------
	\begin{frame}
    \frametitle{Výběr operačního systému}
		\small
		\begin{itemize}
			\item instalace windows již obsahuje všechny knihovny
			\item v případě ubuntu je třeba přidat ppa, aby se stáhla poslední verze KiCAD \kicadVersion\
		\end{itemize}
		
		\begin{center}
			\includegraphics[scale=0.3]{obr/kicad_dwnld.png}
		\end{center}
	\end{frame}
%------------------------------------------------------------------------------
	\begin{frame}
    \frametitle{Stažení instalačního souboru}
		\small
		\begin{itemize}
			\item stáhnout aktuální stabilní verzi \kicadVersion\
		\end{itemize}
		
		\begin{center}
			\includegraphics[scale=0.3]{obr/kicad_stbv.png}
		\end{center}
	\end{frame}
%------------------------------------------------------------------------------
	\begin{frame}
    \frametitle{Instalace - Windows}
    	\textbf{Vhodný návod v podobě videa na youtube:} https://www.youtube.com/watch?v=Cu2VlXy-PzM \\~\\
    	
    	\textbf{Poznámky:}
		\begin{enumerate}
			\item poklepat na stažený instalační soubor
			\item v prvním okně zvolit další,
			\item vše ve volbě součástí nechat zaškrtnuté, jen v případě jazyků zrušit vše kromě češtiny a angličtiny,
			\item zvolit další a přejít do nastavení umístění, umístění doporučuji nechat původní předepsané,
			\item zvolit další a nechat proběhnout instalaci
		\end{enumerate}
	\end{frame}
%------------------------------------------------------------------------------
	\begin{frame}
    \frametitle{První spuštění}
    \small
    	Po instalaci se v nabídce start objeví několik nových programů:
      \begin{tabular}{ m{6cm} m{2cm} }
         \begin{itemize}
           \item \textbf{KiCad}
           \item Eeschema
           \item Pcbnew
           \item Gerbview
           \item PCB calculator
           \item Pagelayout editor
         \end{itemize}
         & 
        \begin{minipage}{\textwidth}
          \includegraphics[scale=0.3]{obr/nabStart.png}
        \end{minipage}
      \end{tabular} 
   
  Vždy spouštíme KiCad, chceme pracovat s projekty.
	\end{frame}
%------------------------------------------------------------------------------
	\begin{frame}
    \frametitle{První spuštění}
    \small
    Při prvním spuštění se program dotáže na tvorbu a umístění souboru \textit{\textbf{sym-lib-table}}. V tomto případě nechte doporučenou první volbu.
    \begin{center}
			\includegraphics[scale=0.4]{obr/kicad_nabidka.png}
		\end{center}
	\end{frame}
	
%------------------------------------------------------------------------------
%Návrh
%------------------------------------------------------------------------------
\section{\texorpdfstring{Návrh}{Navrh}}
\subsection{\texorpdfstring{Nepájivé pole}{Nepajive pole}}
%------------------------------------------------------------------------------
  \begin{frame}
    \frametitle{Nepájivé pole - breadboard}
    \begin{center}
      \includegraphics[width=\textwidth]{obr/breadBoard_bot.png}
    \end{center}
  \end{frame}
%------------------------------------------------------------------------------
  \begin{frame}
    \frametitle{Nepájivé pole - realizace prototypu}
    \begin{center}
      \includegraphics[width=0.8\textwidth]{obr/breadBoard_prot.png}
    \end{center}
  \end{frame}
%------------------------------------------------------------------------------
\subsection{\texorpdfstring{Prototypová deska s prokovy}{Prototypová deska s prokovy}}
%------------------------------------------------------------------------------
  \begin{frame}
    \frametitle{Prototypová deska s prokovy (pady) - perfboard}
    \begin{center}
      \includegraphics[width=0.5\textwidth]{obr/perfBoard_bot.png}
    \end{center}
  \end{frame}
%------------------------------------------------------------------------------
  \begin{frame}
    \frametitle{Prototypová deska s prokovy (pady) - realizace prototypu}
    \begin{center}
      \includegraphics[width=\textwidth]{obr/perfBoard_prot.png}
    \end{center}
  \end{frame}
%------------------------------------------------------------------------------
\subsection{\texorpdfstring{Prototypová deska s pásky}{Prototypova deska s pasky}}
%------------------------------------------------------------------------------
  \begin{frame}
    \frametitle{Prototypová deska s pásky - stripboard}
    \begin{center}
      \includegraphics[width=0.8\textwidth]{obr/stripBoard_bot.png}
    \end{center}
  \end{frame}
%------------------------------------------------------------------------------
  \begin{frame}
    \frametitle{Prototypová deska s pásky - realizace prototypu}
    \begin{center}
      \includegraphics[width=\textwidth]{obr/stripBoard_prot.png}
    \end{center}
  \end{frame}
%------------------------------------------------------------------------------
  \begin{frame}
    \frametitle{Prototypová deska s pásky - návrh}
    \begin{center}
      \includegraphics[width=\textwidth]{obr/stripBoard_desgn.png}
    \end{center}
  \end{frame}
%------------------------------------------------------------------------------
%------------------------------------------------------------------------------
%Součástky
%------------------------------------------------------------------------------
\section{\texorpdfstring{Součástky}{Soucastky}}
%------------------------------------------------------------------------------
  \begin{frame}
    \frametitle{Pasivní prvky}
    \begin{columns}
    
    \column{.5\textwidth}
    \begin{itemize}
      \item \textbf{Rezistory}:
        \begin{itemize}
          \item odpor,
          \item ztrátový výkon,
          \item tolerance.
        \end{itemize}
        
      \item \textbf{Kondenzátory}:
        \begin{itemize}
          \item kapacita,
          \item jmenovité napětí, 
          \item materiál (teplotní závislost),
          \item ztrátový činitel,
          \item frekvenční rozsah použití.
        \end{itemize}
    \end{itemize}
    
    \column{.5\textwidth}
    \begin{itemize}
      \item \textbf{Tlumivky, cívky:}
        \begin{itemize}
          \item indukčnost,
          \item činitel jakosti (parazitní odpor),
          \item jmenovitý proud, 
          \item frekvenční rozsah použití.
        \end{itemize}
    \end{itemize}
    \end{columns}
  \end{frame}
%------------------------------------------------------------------------------
  \begin{frame}
    \frametitle{Pouzdra rezistorů}
    \begin{center}
      \includegraphics[width=0.8\textwidth]{obr/resistor-wattage.png}
    \end{center}
  \end{frame}
%------------------------------------------------------------------------------
  \begin{frame}
    \frametitle{Značení keramických kondenzátorů, třída 2}
    \begin{center}
      \includegraphics[width=\textwidth]{obr/classII.png}
    \end{center}
  \end{frame}
%------------------------------------------------------------------------------
  \begin{frame}
    \frametitle{Příklad - spojování prvků}
    
    \begin{itemize}
      \item Dva rezistory $R_1 = 15 \Omega$, $R_2 = 150 \Omega$  jsou spojeny 
      paralelně, na kterém je vyšší výkonová ztráta?
      \item Dva rezistory $R_1 = 15 \Omega$, $R_2 = 150 \Omega$  jsou spojeny 
      sériově, na kterém je vyšší výkonová ztráta?
      \item Dva kondenzátory $C_1 = 1 nF$, $R_2 = 10 nF$  jsou spojeny 
      sériově, na kterém je vyšší napětí?
    \end{itemize}
    
  \end{frame}
%------------------------------------------------------------------------------
  \begin{frame}
    \frametitle{Aktivní prvky}
    \begin{columns}
    
    \column{.5\textwidth}
    \begin{itemize}
      \item \textbf{Diody}:
        \begin{itemize}
          \item propustný proud,
          \item závěrné napětí,
          \item prahové napětí,
          \item doba závěrného zotavení,
          \item kapacita.
        \end{itemize}
      
      \item \textbf{Integrované obvody:}
        \begin{itemize}
          \item napájecí napětí,
          \item charakteristiky vstupů: napětí, proud, impedance,
          \item charakteristiky výstupů: napětí, proud, typ zátěže, spínací časy...
        \end{itemize}
    \end{itemize}
    
    \column{.5\textwidth}
    \begin{itemize}
      \item \textbf{Tranzistory}:
        \begin{itemize}
          \item proud kolektorem (drainem),
          \item napětí mezi kolektor-emitor (drain-source), 
          \item zesilovací činitel (převodní admitance),
          \item ztrátový výkon,
          \item frekvenční rozsah použití.
        \end{itemize}
    \end{itemize}
    \end{columns}
  \end{frame}
%------------------------------------------------------------------------------
  \begin{frame}
    \frametitle{Příklad - spojování prvků}
    
    \begin{itemize}
      \item Rezistor $R_1 = 1,1 k\Omega$ a dioda 1N4007  jsou spojeny 
      seriově a připjeny ke zdroji napětí 12 V. Jaký proud teče obvodem?
      \item Jaký rezistor se má zvolit do série k LED BL-BD0141 při napájecím
      napětí 5~V pro zvolený proud 10~mA?
      \item Jaký rezistor se má zvolit do série k LED BL-BD0141 při napájecím
      napětí 10~V se střídou 50\% pro zvolený proud 10~mA?
      \item Jaký rezistor se má zvolit do série s bází tranzistoru BC546B, tak
      aby spolehlivě sepnul (saturace) kolektorovou zátěž o velikosti 
      $R = 100\Omega$ v zapojení SE při napájecím napětí 5~V?
    \end{itemize}
    
  \end{frame}
%------------------------------------------------------------------------------
\section{\texorpdfstring{Prezetace zadání}{Prezentace zadani}}
%------------------------------------------------------------------------------
  \begin{frame}
    \frametitle{Hlavolam - automatický zámek}
    \begin{center}
      \includegraphics[width=0.8\textwidth]{obr/hlavolam.png}
    \end{center}
  \end{frame}
%------------------------------------------------------------------------------
  \begin{frame}
    \frametitle{Hlavolam - důležité části}
    \begin{itemize}
      \item Arduino NANO: \href{https://www.gme.cz/klon-arduino-nano-v3-0-r3-ch340g}{arduino},
      \item posuvné registry 74HC595 a 74HC165, návody: \href{https://dronebotworkshop.com/shift-registers/}{registry},
      \item podsvícená tlačítka: \href{https://www.gme.cz/tlacitkovy-spinac-p-pb61412l-508}{spinac},
      \item stabilizátor napětí 5 V, kvůli napájení ze sběrnice zdroje 12~V: \href{https://www.gme.cz/stabilizator-pevneho-napeti-st-microelectronics-thomson-7805cv-stm}{7805},
      \item chladič pro stabilizátor: \href{https://www.gme.cz/v7142a-black}{chladič},
      \item relé pro sepnutí obvodu zámku: \href{https://www.gme.cz/jazyckova-rele-cosmo-reled1a-051-000}{relé},
      \item bez krabičky - provedení do panelu na 4 distanční sloupky.
    \end{itemize}
  \end{frame}
%------------------------------------------------------------------------------
  \begin{frame}
    \frametitle{Hlavolam - co budu ověřovat nebo optimalizovat?}
    \begin{itemize}
      \item celkovou logickou funkci obvodu - program,
      \item vstupní rozsah napájecích napětí 9 - 24 V kvůli oteplení stabilizátoru,
      \item čitelnost světelných tlačítek v různých světelných podmínkách,
      \item zákmity na tlačítkách, dobu reakce mezi stiskem a rozsvícením,
      \item činnost kontaktu relé s elektronickým zámkem (možná přepětí apod.)
    \end{itemize}
  \end{frame}
%------------------------------------------------------------------------------
\end{document}