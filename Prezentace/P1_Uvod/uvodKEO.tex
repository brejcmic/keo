%Beamer class
\documentclass{beamer}

\usepackage[czech]{babel}
\usepackage[utf8]{inputenc}
\usepackage{fontenc}
\usepackage{tgheros}
\usepackage{array}
\usepackage{color}
\usepackage{hyperref}

\usetheme{Antibes}
\usecolortheme{crane}


\title[B0B13KEO]{B0B13KEO}
\subtitle[KEO] {Konstrukce a realizace elektronických obvodů}
\author[Brejcha]{\texorpdfstring{Michal Brejcha\newline\url{brejcmic@fel.cvut.cz}}{Michal Brejcha}}
\institute[CVUT]{ČVUT v Praze, FEL}
\date[Praha, 2020]{Praha, 2020}

%------------------------------------------------------------------------------
%Konstanty a definice
%------------------------------------------------------------------------------
\newtheorem{myDef}{}
\newcommand{\kicadVersion}{5.1.6.}

\begin{document}
%------------------------------------------------------------------------------
%Uvodni slajd
%------------------------------------------------------------------------------
\frame{\titlepage}

\begin{frame}
\frametitle{Obsah} 
\tableofcontents
\end{frame}

\AtBeginSection[]
{
  \begin{frame}
    \frametitle{Téma}
    \tableofcontents[currentsection]
  \end{frame}
}

%------------------------------------------------------------------------------
%Bezpecnost
%------------------------------------------------------------------------------
\section{\texorpdfstring{Bezpečnost}{Bezpecnost}}
%------------------------------------------------------------------------------
	\begin{frame}
    \frametitle{Základní pravidla bezpečnosti}
		\begin{itemize}
		\item Vstup do laboratoří a práce v laboratoři jsou dovoleny jen za přítomnosti učitele.
		\item Manipulace s přístrojovým vybavením laboratoře je dovolena jen v prostorách laboratoře.
		\item Zapínání laboratorních stolů (případně jiných zařízení nn) je dovoleno jen se souhlasem a dohledem učitele.
		\item \textbf{\color{red}{Laboratorní stůl nebo celou laboratoř je dovoleno (jste povinni) kdykoliv vypnout bez výstrahy v případě hrozícího nebezpečí. \uv{BEZPEČNOSTNÍ TLAČÍTKA}}}
		\end{itemize}
	\end{frame}
%------------------------------------------------------------------------------
	\begin{frame}
    \frametitle{Omezení a předpisy}
		\begin{itemize}
		\item V laboratoři není dovolena konzumace potravin,
    \item z laboratoře není dovoleno odnášet jakékoliv přístroje a vlastní přístroje je možné použít (připojit na napájení, měřit s nimi apod.) jen po dohodě s učitelem,
		\item není dovoleno používání mobilních telefonů v průběhu výuky uvnitř laboratoře, pokud se nejedná o případ tísňového volání,
		\item studenti jsou povinni dodržovat zásady protipožární ochrany,
		\item závady na zařízení je nutné ihned hlásit vyučujícímu.
		\end{itemize}
	\end{frame}
%------------------------------------------------------------------------------
	\begin{frame}
    \frametitle{Rizika}
		\begin{itemize}
		\item Úraz elektrickým proudem: práce s nn, přítomnost nekrytých svorek na laboratorním stole.
		\item Popáleniny: páječka - pájení, chybný návrh - horká součástka.
		\item Řezné nebo tržné rány: odizolování vodičů pomocí nože, rozšiřování vrtaných otvorů.
		\item Otrava nebo poleptání chemikáliemi: použití rozpouštědel při mytí pcb, použití chemie při pájení.
		\end{itemize}
	\end{frame}
%------------------------------------------------------------------------------
	\begin{frame}
    \frametitle{Univerzální postup v případě nebezpečí}
		\begin{enumerate}
		\item Zajištění bezpečnosti:\\
		rozpojení elektrického obvodu (bezpečnostní tlačítka), odpojení přítomných přístrojů, uzavření příp. odstranění nebo zabránění šíření (louže - těkavé látky) chemických látek
		\item První pomoc postiženému:\\ chlazení popáleného místa studenou vodou, zastavení krvácení, umělé dýchání, nepřímá srdeční masáž.
		\item Upozornění lektora (zodpovědného pracovníka laboratoří), na vzniklou situaci.
		\item Přivolání lékařské pomoci (tel.: 155), uvědomění vrátnice (tel.: 2222).
		\end{enumerate}
	\end{frame}
%------------------------------------------------------------------------------
	\begin{frame}
    \frametitle{Požární bezpečnost - povinnosti}
		\begin{enumerate}
		\item Počínat si tak, aby nedocházelo ke vzniku požáru, zejména při používání tepelných, elektrických, plynových a jiných spotřebičů, při skladování a používání hořlavých nebo požárně nebezpečných látek, manipulaci s nimi nebo otevřeným ohněm či jiným zdrojem zapálení
		\item Neomezovat přístup k rozvodným zařízením elektrické energie a k uzávěrům vody a topení.
		\end{enumerate}
	\end{frame}
%------------------------------------------------------------------------------
	\begin{frame}
    \frametitle{Požární bezpečnost - zdolávání požáru}
		\begin{enumerate}
		\item Hlasitým opakovaným voláním (\textbf{HOŘÍ!}) vyhlásit požární poplach pro své okolí.
		\item Provést nutná opatření pro záchranu ohrožených osob.
    \item Uhasit požár, je\-li to možné, nebo provést nutná opatření k zamezení jeho šíření.
    \item Ohlásit neodkladně na určeném místě zjištěný požár ev. zabezpečit jeho ohlášení (tel.: 150).
    \item Ostatní osoby opustí spořádaně budovu a soustředí se na shromaždišti. V době požárního poplachu je přísně \textbf{zakázáno používat výtah!}
		\end{enumerate}
	\end{frame}
%------------------------------------------------------------------------------
%Napln cviceni
%------------------------------------------------------------------------------
\section{\texorpdfstring{Náplň cvičení}{Napln Cviceni}}
%------------------------------------------------------------------------------
  \begin{frame}
    \frametitle{Osnova cvičení}
		\begin{enumerate}
          \small
          \item Úvod. Bezpečnostní předpisy. Zadání témat.
          \item Praktické provedení elektronického obvodu nebo jeho části, dislokace součástek, obvod na nepájivém poli.
          \item Prezentace elektronických obvodů zamýšlených k výrobě
          \item Návrh DPS programem KiCAD
          \item Návrh DPS programem KiCAD
          \item Ověření funkce určité části obvodu v laboratoři
          \item Kontrola návrhů, podkladů a generování výrobních dat pro výrobu (\textbf{nutný hotový návrh})
          \item Specifické vlastnosti elektronických součástek
          \item Vrtání DPS, kontrola DPS, úpravy DPS do krabiček
          \item Realizace elektronického obvodu – pájení 
          \item Uvádění elektronického obvodu do provozu
          \item Ověřování funkce obvodu a měření
          \item Závěrečná zpráva
          \item Zápočet
		\end{enumerate}
	
	\end{frame}
%------------------------------------------------------------------------------
	\begin{frame}
    \frametitle{Zápočet}
		
		\begin{itemize}
			\item Předvést funkci výrobku,
			\item odevzdat zprávu o výrobku:
      
      \begin{itemize}
        \item Název výrobku, jméno studenta, datum.
        \item Úplné zadání (funkce, parametry, rozsahy apod.).
        \item Popis funkce, výpočty obvodů, schéma zapojení.
        \item Otisk DPS, osazovací schéma.
        \item Rozpiska součástek.
        \item Výsledky měření.
        \item Zhodnocení.
      \end{itemize}
		\end{itemize}
	
	\end{frame}
%------------------------------------------------------------------------------
	\begin{frame}
    \frametitle{Výroba elektronického obvodu}
		
		\begin{itemize}
			\item DPS vyrábí a platí škola,
			\item součástky kupuje student.
			\item Návrh obvodu lze získat od jiného autora - např. knížka, web...
			\item Pokud již existuje dps, lze ji použít pro inspiraci, nicméně předpokládají se vlastní úpravy řešitele a hlavně její překreslení v návrhovém programu.
		\end{itemize}
	
	\end{frame}
%------------------------------------------------------------------------------
	\begin{frame}
    \frametitle{Vlastnosti zadání}
		
		\begin{itemize}
			\item Obvod s minimálně 30 součástkami,
			\item převážně THT montáž (jednovrstvý plošný spoj nebo dvouvrstvý \textbf{bez prokovů}),
      \item napájení výhradně malým napětím,
      \item vyhýbejte se programovatelným součástkám,
      \item pokud chcete procesor, tak Arduino (snadné ověření funkce obvodu),
      \item jen nízkofrekvenční obvody, 
      \item na relativně malé výkony.
		\end{itemize}
	
	\end{frame}
%------------------------------------------------------------------------------
	\begin{frame}
    \frametitle{Příklad jednoduchého obvodu $>$30 součástek}
		\textbf{Voltmetr Arduino}
		\includegraphics[scale=0.24]{obr/eo_voltSch.png}
	
	\end{frame}
%------------------------------------------------------------------------------
	\begin{frame}
    \frametitle{Příklad jednoduchého obvodu $>$30 součástek}
		\textbf{Měření charakteristické impedance kabelu}
    \begin{center}
      \includegraphics[scale=0.26]{obr/eo_impSch.png}
    \end{center}
	
	\end{frame}
%------------------------------------------------------------------------------
	\begin{frame}
    \frametitle{Příklad jednoduchého obvodu $>$30 součástek}
		\textbf{Signalizace ztráty napájení}
    \begin{center}
      \includegraphics[scale=0.26]{obr/eo_detBrd.png}
    \end{center}
	
	\end{frame}
%------------------------------------------------------------------------------
	\begin{frame}
    \frametitle{Co potřebuji na příští hodinu?}
    
    \begin{enumerate}
      \item Schéma zapojení obvodu, který chci vytvořit (na papíře).
      \item Seznam parametrů obvodu, např.: napájecí napětí, vstupní a výstupní
      impedance, typ zátěže, generované frekvence, atd.
      \item Seznam součástek - GME.
    \end{enumerate}
    
    
    \textbf{Poznámka:} \\
    Minimálně je potřeba mít schéma zapojení obvodu, zbytek můžeme vypracovat na hodině.
	
	\end{frame}
%------------------------------------------------------------------------------
	\begin{frame}
    \frametitle{Co je v nabídce, když nemůžu na nic vlastního přijít?}
    
    Následující témata je si nutné zamluvit a \textbf{obvod se odevzdává} (neplatíte součástky). Obvody budou složit ve cvičení jiných předmětů nebo jiným kolegům.
    
    \begin{enumerate}
      \item Obvod měření charakteristické impedance kabelu.
      \item Napájecí bateriový zdroj 150 V.
      \item Přepínatelná zátěž (50 W).
      \item Budící obvod pro ovládání krokového motorku.
    \end{enumerate}
    
	\end{frame}
%------------------------------------------------------------------------------
%Instalace KiCAD
%------------------------------------------------------------------------------
\section{\texorpdfstring{Instalace KiCAD}{Instalace Kicad}}
%------------------------------------------------------------------------------
	\begin{frame}
    \frametitle{Stažení návrhového systému KiCAD}
		
		\begin{description}
			\item[url:] http://kicad-pcb.org/
			\item[sekce:] download
		\end{description}
		
		\begin{center}
			\includegraphics[scale=0.3]{obr/kicad_url.png}
		\end{center}
	\end{frame}
%------------------------------------------------------------------------------
	\begin{frame}
    \frametitle{Výběr operačního systému}
		\small
		\begin{itemize}
			\item instalace windows již obsahuje všechny knihovny
			\item v případě ubuntu je třeba přidat ppa, aby se stáhla poslední verze KiCAD \kicadVersion\
		\end{itemize}
		
		\begin{center}
			\includegraphics[scale=0.3]{obr/kicad_dwnld.png}
		\end{center}
	\end{frame}
%------------------------------------------------------------------------------
	\begin{frame}
    \frametitle{Stažení instalačního souboru}
		\small
		\begin{itemize}
			\item stáhnout aktuální stabilní verzi \kicadVersion\
		\end{itemize}
		
		\begin{center}
			\includegraphics[scale=0.3]{obr/kicad_stbv.png}
		\end{center}
	\end{frame}
%------------------------------------------------------------------------------
	\begin{frame}
    \frametitle{Instalace - Windows}
    	\textbf{Vhodný návod v podobě videa na youtube:} https://www.youtube.com/watch?v=Cu2VlXy-PzM \\~\\
    	
    	\textbf{Poznámky:}
		\begin{enumerate}
			\item poklepat na stažený instalační soubor
			\item v prvním okně zvolit další,
			\item vše ve volbě součástí nechat zaškrtnuté, jen v případě jazyků zrušit vše kromě češtiny a angličtiny,
			\item zvolit další a přejít do nastavení umístění, umístění doporučuji nechat původní předepsané,
			\item zvolit další a nechat proběhnout instalaci
		\end{enumerate}
	\end{frame}
%------------------------------------------------------------------------------
	\begin{frame}
    \frametitle{První spuštění}
    \small
    	Po instalaci se v nabídce start objeví několik nových programů:
      \begin{tabular}{ m{6cm} m{2cm} }
         \begin{itemize}
           \item \textbf{KiCad}
           \item Eeschema
           \item Pcbnew
           \item Gerbview
           \item PCB calculator
           \item Pagelayout editor
         \end{itemize}
         & 
        \begin{minipage}{\textwidth}
          \includegraphics[scale=0.3]{obr/nabStart.png}
        \end{minipage}
      \end{tabular} 
   
  Vždy spouštíme KiCad, chceme pracovat s projekty.
	\end{frame}
%------------------------------------------------------------------------------
	\begin{frame}
    \frametitle{První spuštění}
    \small
    Při prvním spuštění se program dotáže na tvorbu a umístění souboru \textit{\textbf{sym-lib-table}}. V tomto případě nechte doporučenou první volbu.
    \begin{center}
			\includegraphics[scale=0.4]{obr/kicad_nabidka.png}
		\end{center}
	\end{frame}
%------------------------------------------------------------------------------
\end{document}